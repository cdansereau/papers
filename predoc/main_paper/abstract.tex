\abstract

 %(150 a 250 mots) (1 page)
Functional magnetic resonance imaging (fMRI) gives a non-invasive measure of functional connectivity throughout the brain in individual human subjects. This imaging modality captures a massive amount of connections, in the order of $10^8$, some of which may be biomarkers of Alzheimer's disease (AD). The brain is highly structured into a nested hierarchy of networks, which can be leveraged to reduce the dimensionality of brain connectivity into a limited set of biologically meaningful features. The objective of this project is to develop multiscale clustering techniques for feature extraction, in conjunction with random forest for prediction of clinical diagnosis and prognosis of AD, using fMRI connectivity. I will investigate the case of multicentric fMRI data including a few hundreds of participants, i.e. the AD neuroimaging initiative (ADNI) sample, which is the typical design found in phase II pharmaceutical clinical trials. Site-specific MRI set-ups may bias the fMRI measures, and I am thus developing procedures for inter-site normalization. The main outcome of this project will be a prediction pipeline for the data-driven identification of AD biomarkers in resting-state fMRI.



%(max. of 10, not words in the title)
{\bfseries Keywords: fMRI, Alzeimer's disease, biomarker, multisite.}
